\documentclass[a4paper,11pt]{article}
\usepackage[UTF8]{ctex} % 支持中文
\usepackage[margin=1in]{geometry}
\usepackage{hyperref}
\usepackage{xcolor}
\usepackage{listings}
\usepackage{amsmath}
\usepackage{tcolorbox}
\usepackage{booktabs}
\usepackage{graphicx}

% 定义代码块样式
\lstset{
    language=SQL,
    basicstyle=\ttfamily\small,
    keywordstyle=\color{blue}\bfseries,
    commentstyle=\color{gray},
    stringstyle=\color{red},
    frame=single,
    breaklines=true,
    numbers=left,
    numberstyle=\tiny\color{gray},
    showstringspaces=false
}

% 定义提示框样式
\newtcolorbox{warningbox}{
    colback=red!5!white,
    colframe=red!75!black,
    title=\textbf{⚠️ 易错点 / Pitfall},
    fonttitle=\bfseries
}

\newtcolorbox{correctionbox}{
    colback=yellow!10!white,
    colframe=orange!85!black,
    title=\textbf{💡 修正与现代观点 / Modern Practice},
    fonttitle=\bfseries
}

\title{\textbf{CS213 数据库系统原理 (H) 详细Lecture Note}}
\author{基于 Shiqi Yu (SUSTech) 讲义整理}
\date{\today}

\begin{document}

\maketitle
\tableofcontents
\newpage

\section{Chapter 1: 数据库系统导论 (Introduction to Databases)}

\subsection{1.1 数据库系统的本质}
数据库系统不仅仅是数据的集合,它是一个负责管理大量、复杂、相关联数据的软件系统。
\begin{itemize}
    \item \textbf{组成要素}:
    \begin{itemize}
        \item \textbf{数据 (Data)}:相关联的数据集合。
        \item \textbf{程序 (Programs)}:一组用于访问和操作数据的程序。
        \item \textbf{环境 (Environment)}:一个既方便又高效的使用环境。
    \end{itemize}
    \item \textbf{历史演变}:
    \begin{itemize}
        \item \textbf{早期 (File Systems)}:数据直接存储在文件系统中。存在数据冗余、不一致、访问困难、原子性缺失、并发访问困难及安全问题。
        \item \textbf{1970年}:Edgar F. Codd 发表了划时代的论文 \textit{"A Relational Model of Data for Large Shared Data Banks"},提出了关系模型。
        \item \textbf{1980s}:SQL 成为工业标准;关系型数据库 (RDBMS) 商业化(如 Oracle, DB2)。
        \item \textbf{2000s}:大数据时代,NoSQL 系统(如 Google BigTable, Amazon Dynamo)兴起,用于处理非结构化数据和超大规模数据;MapReduce 流行。
        \item \textbf{2010s}:NewSQL 出现,试图结合 NoSQL 的可扩展性和 SQL 的 ACID 特性;多核内存数据库成为趋势。
    \end{itemize}
\end{itemize}

\subsection{1.2 关系模型 (Relational Model)}
\begin{itemize}
    \item \textbf{关系 (Relation)}:即“表”(Table)。
    \item \textbf{元组 (Tuple)}:表中的一行 (Row/Record),代表一个实体或事实。
    \item \textbf{属性 (Attribute)}:表中的一列 (Column),代表数据的一个特征。
    \item \textbf{核心思想}:表中的一行数据是相互关联的 (related),即“关系”一词的由来。
    \item \textbf{集合论基础}:关系是数学上的集合,因此可以对表进行集合运算(选择、投影、连接等)从而产生新的表。
\end{itemize}

\subsection{1.3 键 (Keys)}
在关系模型中,\textbf{不允许存在重复的行} (No duplicates)。为了区分不同的行,需要使用“键”。
\begin{itemize}
    \item \textbf{超键 (Super Key)}:能唯一标识一个元组的一个或多个属性的集合。
    \item \textbf{候选键 (Candidate Key)}:最小的超键(没有多余属性)。
    \item \textbf{主键 (Primary Key, PK)}:被数据库设计者选中用来作为主要标识符的候选键。通常选择最简单、最稳定的那个。
    \end{itemize}

\begin{warningbox}
虽然理论上关系集合不允许重复行,但在实际的 SQL 数据库实现中,如果没有显式定义主键或唯一约束,表是可以包含完全重复的行的(这被称为 bag 而不是 set)。这在数据处理中通常是不良设计。
\end{warningbox}

\subsection{1.4 范式 (Normalization)}
范式化是组织数据以减少冗余和依赖的过程。
\begin{itemize}
    \item \textbf{第一范式 (1NF)}:属性必须是原子的 (Atomic)。
    \begin{itemize}
        \item 错误示例:一个单元格内存放 "Zhang San, Li Si"。
        \item 正确做法:一行只存一个名字,或者分拆成多行。
    \end{itemize}
    \item \textbf{第二范式 (2NF)}:满足 1NF,且非主属性必须\textbf{完全依赖}于主键(消除部分依赖)。
    \begin{itemize}
        \item 如果主键是 (StudentId, CourseId),那么 "StudentName" 不应该放在这张表里,因为它只依赖于 StudentId,而不依赖于 CourseId。
    \end{itemize}
    \item \textbf{第三范式 (3NF)}:满足 2NF,且非主属性不依赖于其他非主属性(消除传递依赖)。
    \begin{itemize}
        \item 例如:(MovieId, DirectorId, DirectorName)。DirectorName 依赖于 DirectorId,而 DirectorId 依赖于 MovieId。这构成了传递依赖,应该拆分为两张表。
    \end{itemize}
\end{itemize}

\section{Chapter 2: SQL 基础 (Introduction to SQL)}

\subsection{2.1 SQL 语言概览}
SQL (Structured Query Language) 起源于 IBM 的 SEQUEL 项目,旨在提供一种类似英语的结构化查询语言。
\begin{itemize}
    \item \textbf{DDL (Data Definition Language)}:定义数据结构。核心指令:\texttt{CREATE}, \texttt{ALTER}, \texttt{DROP}。
    \item \textbf{DML (Data Manipulation Language)}:操作数据内容。核心指令:\texttt{INSERT}, \texttt{UPDATE}, \texttt{DELETE}, \texttt{SELECT}。
\end{itemize}

\begin{correctionbox}
SQL 是一种声明式语言 (Declarative),你告诉数据库\textbf{你想要什么 (What)},而不是\textbf{怎么做 (How)}。但在性能调优时,理解底层的“怎么做”(执行计划)至关重要。
\end{correctionbox}

\subsection{2.2 创建表 (CREATE TABLE)}
\begin{lstlisting}
CREATE TABLE table_name (
    column1 datatype [constraints],
    column2 datatype [constraints],
    ...
);
\end{lstlisting}
\begin{itemize}
    \item \textbf{标识符规则}:表名和列名通常不区分大小写(Case-insensitive)。建议使用下划线命名法(snake\_case),避免使用空格或保留字。
    \item \textbf{数据类型 (Data Types)}:
    \begin{itemize}
        \item \textbf{文本}:\texttt{CHAR(n)} (定长,不足补空格), \texttt{VARCHAR(n)} (变长), \texttt{CLOB/TEXT} (大文本)。
        \begin{warningbox}
        Oracle 中 \texttt{VARCHAR2} 是推荐类型,空字符串 \texttt{''} 在 Oracle 中被视为 \texttt{NULL},这与其他数据库不同。
        \end{warningbox}
        \item \textbf{数字}:\texttt{INT}, \texttt{FLOAT}, \texttt{NUMERIC(p,s)} / \texttt{DECIMAL(p,s)} (p为总位数,s为小数位)。
        \item \textbf{日期}:\texttt{DATE} (通常含时间,Oracle/MySQL不同), \texttt{TIMESTAMP} (更高精度), \texttt{DATETIME} (SQL Server)。
    \end{itemize}
\end{itemize}

\subsection{2.3 约束 (Constraints)}
约束是保证数据一致性和正确性的关键机制。
\begin{itemize}
    \item \textbf{NOT NULL}:列值不能为空。
    \item \textbf{PRIMARY KEY}:唯一标识行,隐含 NOT NULL 和 UNIQUE。
    \item \textbf{UNIQUE}:列值必须唯一(但通常允许有多个 NULL,视 DBMS 而定)。
    \item \textbf{CHECK}:自定义逻辑校验(例如 \texttt{year > 1900})。
    \begin{correctionbox}
    MySQL 在 8.0.16 版本之前会忽略 CHECK 约束,这在当时是一个主要缺陷。现在的版本已支持。
    \end{correctionbox}
    \item \textbf{FOREIGN KEY (Referential Integrity)}:确保引用的值在父表中存在。
    \begin{itemize}
        \item 语法:\texttt{FOREIGN KEY (col) REFERENCES other\_table(col)}。
        \item 作用:防止出现“孤儿数据”。删除父表数据时,如果子表有引用,数据库会报错或级联删除(需配置)。
    \end{itemize}
\end{itemize}

\subsection{2.4 插入数据 (INSERT)}
\begin{lstlisting}
INSERT INTO table_name (col1, col2) VALUES (val1, val2);
\end{lstlisting}
\begin{itemize}
    \item \textbf{字符串引用}:标准 SQL 使用\textbf{单引号} \texttt{'string'}。
    \item \textbf{转义引号}:在字符串中表示单引号,通常使用双写单引号 \texttt{'It''s ok'}。
    \item \textbf{日期输入}:强烈建议使用格式转换函数(如 \texttt{TO\_DATE} 或 \texttt{DATE()})显式输入日期,不要依赖隐式转换,因为不同地区的日期格式(DD/MM/YYYY vs MM/DD/YYYY)极易混淆。
\end{itemize}

\section{Chapter 3: 单表查询 (Retrieving Data from One Table)}

\subsection{3.1 SELECT 基础}
\begin{itemize}
    \item \texttt{SELECT *}:检索所有列。
    \begin{warningbox}
    \textbf{程序开发禁忌}:在生产代码中尽量避免使用 \texttt{SELECT *}。如果表结构变更(增加列),可能会导致程序崩溃或不仅浪费网络带宽。应显式列出需要的列名。
    \end{warningbox}
    \item \textbf{投影 (Projection)}:选择特定的列。
    \item \textbf{选择 (Selection/Restriction)}:使用 \texttt{WHERE} 子句筛选行。
\end{itemize}

\subsection{3.2 过滤与运算符 (WHERE Clause)}
\begin{itemize}
    \item \textbf{比较符}:\texttt{=}, \texttt{<>} (或 \texttt{!=}), \texttt{<}, \texttt{>}, \texttt{<=}, \texttt{>=}。
    \item \textbf{逻辑符}:\texttt{AND}, \texttt{OR}, \texttt{NOT}。
    \begin{itemize}
        \item \textbf{优先级}:\texttt{AND} 的优先级高于 \texttt{OR}。混合使用时务必使用括号 \texttt{()} 明确意图。
    \end{itemize}
    \item \textbf{范围与列表}:\texttt{BETWEEN a AND b} (包含边界), \texttt{IN (a, b, c)}。
    \item \textbf{模糊匹配}:\texttt{LIKE}。
    \begin{itemize}
        \item \texttt{\%}:匹配任意长度字符(包括0个)。
        \item \texttt{\_}:匹配任意单个字符。
        \item 大小写敏感性取决于 DBMS 和 Collation 设置。
    \end{itemize}
\end{itemize}

\subsection{3.3 NULL 的陷阱}
\begin{warningbox}
\textbf{NULL 不等于 NULL}。
\end{warningbox}
\begin{itemize}
    \item NULL 代表“未知”或“不存在”。
    \item 任何与 NULL 进行的比较运算(\texttt{=}, \texttt{!=}, \texttt{>}, \texttt{<})结果都是 \textbf{Unknown}(在 SQL 中会被视为 False)。
    \item \textbf{错误写法}:\texttt{WHERE col = NULL} (永远不返回任何行)。
    \item \textbf{正确写法}:\texttt{WHERE col IS NULL} 或 \texttt{WHERE col IS NOT NULL}。
    \item \textbf{三值逻辑}:TRUE, FALSE, UNKNOWN。注意 \texttt{NOT UNKNOWN} 仍然是 \texttt{UNKNOWN}。
\end{itemize}

\subsection{3.4 函数与转换}
\begin{itemize}
    \item \textbf{字符串}:\texttt{UPPER()}, \texttt{LOWER()}, \texttt{SUBSTR()}, 连接符 \texttt{||} (ANSI) 或 \texttt{+} (SQL Server) 或 \texttt{CONCAT()}。
    \item \textbf{日期}:日期运算极其依赖 DBMS。
    \begin{itemize}
        \item \texttt{DATEADD}, \texttt{DATEDIFF} (SQL Server)。
        \item \texttt{date + interval '1' day} (PostgreSQL/MySQL)。
        \item 比较 \texttt{DATETIME} 和 \texttt{DATE} 时要注意时间部分可能导致相等判断失败。
    \end{itemize}
    \item \textbf{类型转换}:\texttt{CAST(col AS type)} 是标准语法。隐式转换(如字符串与数字比较)可能导致性能问题(索引失效)。
\end{itemize}

\subsection{3.5 CASE 表达式}
用于在 SQL 中实现 if-else 逻辑。
\begin{lstlisting}
CASE 
    WHEN condition1 THEN result1
    WHEN condition2 THEN result2
    ELSE result3
END
\end{lstlisting}
或者:
\begin{lstlisting}
CASE column
    WHEN val1 THEN res1
    ELSE res2
END
\end{lstlisting}
\textbf{注意}:如果省略 \texttt{ELSE} 且没有匹配项,默认返回 \texttt{NULL}。

\section{Chapter 4: 聚合与连接 (Aggregation and Joins)}

\subsection{4.1 去重 (DISTINCT)}
\begin{itemize}
    \item \texttt{SELECT DISTINCT col1, col2 ...}:去除结果集中所有指定列组合完全相同的重复行。
    \item 这会将结果集转化为真正意义上的“关系”(集合)。
\end{itemize}

\subsection{4.2 聚合函数 (Aggregate Functions)}
\begin{itemize}
    \item \textbf{常见函数}:\texttt{COUNT(*)} (统计所有行), \texttt{COUNT(col)} (统计非NULL值), \texttt{SUM()}, \texttt{AVG()}, \texttt{MIN()}, \texttt{MAX()}。
    \item \textbf{NULL处理}:聚合函数(除 \texttt{COUNT(*)} 外)会自动忽略 NULL 值。
    \item \textbf{GROUP BY}:将数据分组,对每组应用聚合函数。
    \begin{warningbox}
    \textbf{SELECT 列表限制}:在使用 \texttt{GROUP BY} 时,SELECT 子句中只能包含聚合函数或 \texttt{GROUP BY} 中列出的列。不能选择非分组列,因为它们的值是不确定的。
    \end{warningbox}
    \item \textbf{HAVING}:对\textbf{分组后}的结果进行过滤(例如:筛选出平均分大于60的班级)。\texttt{WHERE} 是在分组前过滤。
\end{itemize}

\subsection{4.3 连接 (Joins)}
连接是将多张表的数据结合起来的核心操作。
\begin{itemize}
    \item \textbf{内连接 (INNER JOIN)}:仅返回两张表中满足连接条件的行。
    \item \textbf{语法演变}:
    \begin{correctionbox}
    \textbf{显式 JOIN (ANSI-92) vs 隐式 JOIN (ANSI-89)}:
    讲义中提到了使用 \texttt{FROM A, B WHERE A.id = B.id} 的写法。
    \textbf{现代最佳实践}:强烈建议使用 \texttt{FROM A JOIN B ON A.id = B.id}。
    原因:
    1. 分离了连接逻辑 (ON) 和过滤逻辑 (WHERE),可读性更高。
    2. 防止意外遗漏 WHERE 条件导致笛卡尔积 (Cartesian Product),造成性能灾难。
    \end{correctionbox}
    \item \textbf{多表连接}:可以链式进行 \texttt{JOIN ... ON ... JOIN ... ON ...}。
\end{itemize}

\section{Chapter 5: 高级连接与子查询 (Advanced Joins and Subqueries)}

\subsection{5.1 外连接 (Outer Joins)}
\begin{itemize}
    \item \textbf{LEFT OUTER JOIN}:返回左表的所有行。如果右表没有匹配,右表的列显示为 NULL。
    \item \textbf{RIGHT OUTER JOIN}:返回右表的所有行(不常用,通常习惯改写为 LEFT JOIN)。
    \item \textbf{FULL OUTER JOIN}:返回两张表的所有行,任一边不匹配则补 NULL(MySQL 不直接支持,需用 UNION 模拟)。
    \item \textbf{陷阱}:如果在 \texttt{WHERE} 子句中对右表(Left Join 的被连接表)的列加条件,会将外连接“退化”为内连接。
    \begin{itemize}
        \item 错误:\texttt{... LEFT JOIN B ON ... WHERE B.col = 1}
        \item 正确:\texttt{... LEFT JOIN B ON ... AND B.col = 1} (如果是连接条件)或检查 \texttt{B.col IS NULL}。
    \end{itemize}
\end{itemize}

\subsection{5.2 集合操作 (Set Operators)}
要求两个查询结果的列数相同,且对应列的数据类型兼容。
\begin{itemize}
    \item \textbf{UNION}:合并结果并\textbf{去重}。
    \item \textbf{UNION ALL}:合并结果\textbf{不去重}。效率通常比 UNION 高,因为不需要排序去重。如果确定无重复或需要保留重复,优先使用此项。
    \item \textbf{INTERSECT}:交集(两边都有的行)。
    \item \textbf{EXCEPT / MINUS}:差集(左边有但右边没有的行)。
\end{itemize}

\subsection{5.3 子查询 (Subqueries)}
\begin{itemize}
    \item \textbf{标量子查询}:返回单一值的子查询,可用在 \texttt{SELECT} 列表或 \texttt{WHERE} 比较中。
    \item \textbf{IN / NOT IN}:判断值是否在列表中。
    \begin{warningbox}
    \textbf{NOT IN 与 NULL}:如果子查询结果中包含任何 NULL 值,\texttt{NOT IN} 将永远返回空结果(因为 \texttt{x <> NULL} 是 Unknown)。在这种情况下应优先使用 \texttt{NOT EXISTS}。
    \end{warningbox}
    \item \textbf{EXISTS / NOT EXISTS}:检查子查询是否返回行。通常比 \texttt{IN} 性能更好,且对 NULL 处理更直观。
    \item \textbf{相关子查询 (Correlated Subquery)}:子查询内部引用了外部查询的列。对每一行外部数据,子查询都要执行一次(但在现代优化器下往往会被优化为 Join)。
\end{itemize}

\section{Chapter 6: 排序与窗口函数 (Ordering and Window Functions)}

\subsection{6.1 排序 (ORDER BY)}
\begin{itemize}
    \item \texttt{ORDER BY col [ASC|DESC]}。
    \item 多列排序:\texttt{ORDER BY col1 ASC, col2 DESC}。
    \item \textbf{NULL 排序}:不同 DBMS 处理不同。Oracle 默认 NULL 最大(排最后),SQL Server 默认 NULL 最小。可使用 \texttt{NULLS FIRST/LAST} 语法(如果支持)。
    \item \textbf{分页}:
    \begin{itemize}
        \item MySQL/PG: \texttt{LIMIT x OFFSET y}
        \item SQL Server: \texttt{TOP x} 或 \texttt{OFFSET ... FETCH NEXT ...}
        \item Oracle: \texttt{ROWNUM} (旧) 或 \texttt{FETCH FIRST ...} (12c+)
    \end{itemize}
\end{itemize}

\subsection{6.2 窗口函数 (Window Functions)}
用于在不聚合行的情况下计算聚合值(如“每行的累计总和”、“每行所属组的平均值”)。
\begin{itemize}
    \item 语法:\texttt{Function() OVER (PARTITION BY ... ORDER BY ...)}
    \item \textbf{排名函数}:
    \begin{itemize}
        \item \texttt{ROW\_NUMBER()}:1, 2, 3, 4(即使值相同也强行排序,无并列)。
        \item \texttt{RANK()}:1, 2, 2, 4(有并列,排名跳跃)。
        \item \texttt{DENSE\_RANK()}:1, 2, 2, 3(有并列,排名连续)。
    \end{itemize}
    \item \textbf{聚合窗口}:如 \texttt{SUM(salary) OVER (PARTITION BY dept)} 计算部门总薪资,附加在每一名员工记录后。
\end{itemize}

\section{Chapter 7: 模糊搜索与事务 (Fuzzy Search and Transactions)}

\subsection{7.1 模糊搜索}
\begin{itemize}
    \item \textbf{问题}:Typos(拼写错误),不同语言的拼写差异(Shenzhen vs Shenzen)。
    \item \textbf{解决方案}:
    \begin{itemize}
        \item \texttt{LIKE}:简单的模式匹配,但无法处理拼写错误,且 \texttt{\%abc} 无法利用索引。
        \item \textbf{Soundex}:基于发音的算法(如将姓名转换为发音代码比较)。
        \item \textbf{全文检索 (Full-text Search)}:分词、倒排索引。
        \item \textbf{自定义距离}:如 Levenshtein Distance(编辑距离)。
    \end{itemize}
\end{itemize}

\subsection{7.2 事务 (Transactions)}
事务是逻辑上的一组操作,要么全做,要么全不做。
\begin{itemize}
    \item \textbf{ACID 属性}:
    \begin{itemize}
        \item \textbf{A (Atomicity)}:原子性,不可分割。
        \item \textbf{C (Consistency)}:一致性,事务前后数据需满足约束。
        \item \textbf{I (Isolation)}:隔离性,并发事务互不干扰(存在隔离级别)。
        \item \textbf{D (Durability)}:持久性,提交后数据不丢失。
    \end{itemize}
    \item \textbf{控制语句}:\texttt{BEGIN} (或 \texttt{START TRANSACTION}), \texttt{COMMIT}, \texttt{ROLLBACK}。
    \item \textbf{Auto-commit}:很多数据库默认每条语句就是一个事务,需显式关闭或显式开启事务块。
\end{itemize}

\subsection{7.3 数据加载 (Bulk Loading)}
\begin{itemize}
    \item 逐条 \texttt{INSERT} 效率极低。
    \item 应使用批量加载工具(如 MySQL \texttt{LOAD DATA INFILE}, Oracle \texttt{SQL*Loader}, Postgres \texttt{COPY})。
    \item \textbf{主键生成}:
    \begin{itemize}
        \item \textbf{Sequence} (Oracle, PG):独立对象生成数字。
        \item \textbf{Auto Increment / Identity} (MySQL, SQL Server):列属性自增。
    \end{itemize}
\end{itemize}

\section{Chapter 8: 数据修改与函数 (Update, Delete, Functions)}

\subsection{8.1 UPDATE 与 DELETE}
\begin{itemize}
    \item \textbf{UPDATE}:
    \begin{itemize}
        \item \texttt{UPDATE table SET col=val WHERE ...}
        \item \textbf{致命错误}:忘记写 \texttt{WHERE} 子句会导致全表更新。
        \item 使用子查询更新:\texttt{SET col = (SELECT ...)}。需确保子查询只返回一行,否则报错。
    \end{itemize}
    \item \textbf{DELETE}:
    \begin{itemize}
        \item \texttt{DELETE FROM table WHERE ...}
        \item 逐行删除,记录日志,可回滚。
    \end{itemize}
    \item \textbf{TRUNCATE}:
    \begin{itemize}
        \item DDL 操作,快速清空全表,通常不可回滚(视 DBMS 实现),重置自增 ID。
    \end{itemize}
    \item \textbf{MERGE (Upsert)}:
    \begin{itemize}
        \item 标准语法 \texttt{MERGE INTO target USING source ON match ... WHEN MATCHED UPDATE ... WHEN NOT MATCHED INSERT ...}。
        \item MySQL 使用 \texttt{INSERT ... ON DUPLICATE KEY UPDATE}。
    \end{itemize}
\end{itemize}

\subsection{8.2 存储函数与过程 (Functions \& Procedures)}
\begin{itemize}
    \item \textbf{函数 (Function)}:必须有返回值,通常可用在 SQL 语句中(如 \texttt{SELECT my\_func(col)})。
    \item \textbf{过程 (Procedure)}:可以没有返回值,通过 \texttt{CALL} 或 \texttt{EXECUTE} 调用。
    \item \textbf{N+1 问题}:在 SELECT 列表中调用执行查询的函数,会导致每一行都执行一次查询,性能极差。应改写为 Join。
\end{itemize}

\section{Chapter 9: 过程与触发器 (Procedures and Triggers)}

\subsection{9.1 存储过程优势}
\begin{itemize}
    \item \textbf{性能}:减少网络交互(一次调用,服务器端执行多步)。
    \item \textbf{安全}:可以只授予用户执行存储过程的权限,而不授予直接修改表的权限。
    \item \textbf{逻辑封装}:业务逻辑集中在数据库层。
\end{itemize}

\subsection{9.2 触发器 (Triggers)}
在特定的数据库事件(INSERT/UPDATE/DELETE)发生之前(BEFORE)或之后(AFTER)自动执行的代码。
\begin{itemize}
    \item \textbf{用途}:
    \begin{itemize}
        \item 强制复杂的完整性约束。
        \item 审计(Auditing):记录谁修改了数据。
        \item 自动更新冗余数据(如更新文章表时自动更新分类表的文章计数)。
    \end{itemize}
    \item \textbf{风险}:
    \begin{itemize}
        \item \textbf{隐藏逻辑}:开发者可能不知道数据被触发器修改了。
        \item \textbf{性能杀手}:大量行级触发器(FOR EACH ROW)会显著拖慢写入速度。
        \item \textbf{死循环}:A表触发器更新B表,B表触发器更新A表。
    \end{itemize}
    \item \textbf{建议}:尽量少用,除非无法通过约束或应用层逻辑解决。
\end{itemize}

\section{Chapter 10: 存储与性能 (Storage and Performance)}

\subsection{10.1 物理存储结构}
\begin{itemize}
    \item \textbf{存储金字塔}:寄存器 > 缓存 > 内存 > SSD > HDD > 磁带。速度越快,价格越高,容量越小。
    \item \textbf{RAID}:磁盘阵列。
    \begin{itemize}
        \item RAID 0:条带化,速度快,无冗余(坏一块盘全丢)。
        \item RAID 1:镜像,安全,空间利用率50\%。
        \item RAID 5/10:平衡方案。
    \end{itemize}
    \item \textbf{块/页 (Block/Page)}:数据库I/O的最小单位(通常 4KB, 8KB, 16KB)。读取一行数据实际上通常会读取包含该行的整个页。
\end{itemize}

\subsection{10.2 索引 (Indexing)}
索引是加速数据检索的数据结构(类似于书的目录)。
\begin{itemize}
    \item \textbf{B-Tree (Balanced Tree)}:最常用的索引结构。保持排序,适合范围查找 (\texttt{>, <, BETWEEN}) 和等值查找。
    \item \textbf{索引失效 (Pitfalls)}:
    \begin{warningbox}
    以下情况可能导致索引失效(全表扫描):
    1. \textbf{对索引列使用函数}:\texttt{WHERE YEAR(date\_col) = 2020}。应改为 \texttt{WHERE date\_col BETWEEN '2020-01-01' AND '2020-12-31'}。
    2. \textbf{隐式类型转换}:字符串列存数字,查询时用数字类型比较。
    3. \textbf{前导模糊查询}:\texttt{LIKE '\%abc'}。
    4. \textbf{联合索引不满足最左前缀原则}:索引是 (A, B),查询条件只有 B。
    \end{warningbox}
    \item \textbf{选择性 (Selectivity)}:索引在“区分度高”的列上最有效(如身份证号)。在“性别”这种只有两个值的列上建立索引通常无意义。
    \item \textbf{代价}:加快了读取,减慢了写入(INSERT/UPDATE/DELETE 需要维护索引树)。
\end{itemize}

\subsection{10.3 执行计划 (Execution Plan)}
\begin{itemize}
    \item 使用 \texttt{EXPLAIN} 命令查看数据库如何执行 SQL。
    \item 关键看点:是否使用了索引 (Index Scan/Seek) 还是 全表扫描 (Full Table Scan/Seq Scan)。
\end{itemize}

\end{document}